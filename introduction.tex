\chapter{Introduction}

Based on a rough estimate, there are about $10,000$ trillion ants, $7.6$ billion humans, and a few million elephants in the world.  When considering this small data set, one might come to the conclusion that there are more smaller objects in the world than there are larger objects.  This conclusion not only generally holds true on earth, but it also holds truth in our Universe.  In our solar system, there is one star, several planets, and an almost incomprehensible number of small rocks traveling through space. 

Similarly to distances in space, velocities of objects in space are of higher orders of magnitude than those observed on earth.  For example, our earth travels at approximately $30$ kilometers per second while small rocks can travel anywhere between $11$ to $70$ kilometers per second.  The speeds of these rocks exceed the initial velocity of a bullet.  When we experience a meteor shower, we are witnessing a barrage of these bullet-like rocks.  Fortunately for mankind, earth’s atmosphere provides a protective shield consisting of a condensed array of particles.  Condensed is a term being used in relativity to the vacuum of space in which these rocks spend the majority of their lifetime. 

As a rocky object travels through earth’s atmosphere, it collides with particles which in turn burn up the object in a phenomena known as ablation.  The result is release of energy in the form of both heat and light.  The objects, which ignite in a fiery ball are called fireballs.  For large enough fireballs, the light produced in ablation can be seen from the human eye as shooting stars.

By observing the photometric (light) magnitude, duration, and other properties of individual fireball events, observers can estimate initial energies and sizes of these near-Earth rocky objects.  Given a large enough sample size, observers can also determine the flux, or the number of events within a specific area per time, of fireballs of varying sizes and energies.

While cameras set up by organizations such as NASA and the Spanish Meteor Network (SPMN) provide useful and precise measurements of fireball events, they lack flexibility and affordability.  Often rendered immobile due to their connection to powerful computers, these systems provide only a small piece to the puzzle of fireball observation.  Any individual camera system can only observe up to around 0.03 percent of earth’s total sky.  

The following project aims to analyze the feasibility of the Willamette D6 AllSky camera, a new alternative in fireball research.  Occupying about the same space as a traffic cone, this camera is easily transportable and can be replicated at a fraction of the price of the more expensive professionally used systems.  By comparing flux rates measured through the analysis of data taken from Willamette’s D6 AllSky camera to more well-recognized systems, we will determine if our setup is a practical option for amateur astronomers interested in contributing to fireball research. 

Chapter 2 (the Background section) details useful information surrounding fireballs, their importance, existing research, and the analysis chain that will lead us to the final goal of attaining flux rates.