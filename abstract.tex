\begin{flushleft}

\textbf{\textsc{\LARGE General Abstract}}

\vspace{0.2 in}

As fireballs, more commonly known as shooting stars, fly through Earth’s atmosphere at breakneck speeds, they emit light that can be observed on our planet’s surface. Willamette’s D6 AllSky Survey is a camera system that probes the night’s sky for these events which can be further utilized for large-scale analysis such as determining flux rates. Flux rates depict the number of events that occur per unit time per unit area. Because one camera system can only observe $<0.4\%$ of Earth’s total sky area, amateur astronomers hold a significant role in fireball observations. Their observations provide a robust data sample size that can then be used to gain a deeper understanding of the flux rates and property distributions of fireballs. While complex multi-camera professional systems currently exist, there is need for more economic, accessible, and versatile systems. We will discuss the feasibility of our current observational setup and how it compares to more elaborate existing systems.
\vspace{0.25 in}

\textbf{\textsc{\LARGE Technical Abstract}}

\vspace{0.2 in}

Fireballs, more technically known as bolides, are recognizable by the light they emit through ablation.  The D6 AllSky Camera was designed by Dr. Jed Rembold and Kyle McSwain as an alternative observation system for fireball research.  It is smaller, more portable, and significantly cheaper than most existing systems used by professional astronomers.  By measuring distributions of fireballs, primarily in the form of average flux, we aim to assess the feasibility of using the D6 AllSky Camera as opposed to other systems.  Due to poor weather and other unforeseen complications, our data sample was not sufficiently large enough to produce a confident average flux rate.  However the versatile framework established in this research will assuredly aid in the analysis of fireballs throughout further research.

\end{flushleft}